\chapter{Summary and outlook}\label{ch:summary}

The aim of this thesis was to investigate
the contributions of induced three-body operators
in the IMSRG.
To this end,
we studied the IMSRG(3) truncation
of the IMSRG.
The uncoupled ($m$-scheme) implementation of the IMSRG(3)
is too expensive for nuclear applications,
so we exploited the spherical symmetry
of the IMSRG
when applied to closed-shell nuclei
to arrive at the $J$-scheme IMSRG(3)
fundamental commutators.
We implemented an IMSRG(2)/(3) solver
using these fundamental commutators
that is more performant than
the previous $m$-scheme implementation.
With this implementation in hand,
we were able to do first explorations
of the effects of approximations to the IMSRG(3) truncation
when compared with the IMSRG(2) truncation in ${}^{4}\text{He}$
and ${}^{16}\text{O}$.

To allow for the exploration of different single-particle bases,
Hamiltonians, and oscillator frequencies,
we introduced several approximate IMSRG(3) truncation schemes
that leave out the most expensive fundamental commutators,
one of which was the IMSRG(3)-$N^7$.
We find that in ${}^{4}\text{He}$,
the IMSRG(3)-$N^7$ truncation
offers a small correction at the sub-percent level
to the IMSRG(2) result
for the ground-state energy
in the $\emax=2$ model space considered.
The corrections to the charge radius are larger,
on the order of a few percent.
We see that the IMSRG(3)-$N^7$
performs somewhat better than the IMSRG(2)
away from the optimal HF oscillator frequency,
a trend that needs to be verified
for larger model spaces and different systems.

For ${}^{16}\text{O}$,
we considered
various approximate IMSRG(3) truncation schemes.
These schemes were set up to include
one or several fundamental commutators
on top of the previous truncation scheme,
starting from the IMSRG(2).
The organization of these schemes was
partially physically motivated
(in particular the IMSRG(3)-A scheme),
but most commutator inclusions
were organized by computational cost.
We found that this organization
did not deliver the desired results.
While the IMSRG(3)-A truncation
provided a relatively significant shift
from the IMSRG(2) results,
the inclusion of the remaining
$\mathcal{O}(N^6)$ and
$\mathcal{O}(N^7)$ commutators
provided only small corrections
to the ground-state energy of ${}^{16}\text{O}$.
The final truncation scheme we considered,
which included the first $\mathcal{O}(N^8)$
commutator,
produced another large shift over the previous truncation.
This suggests that computational cost
is not a good proxy for importance in the formation of
an approximate IMSRG(3) truncation.

Looking to the future,
there are clear objectives for our IMSRG(3) implementation.
The first is tuning the performance of the implementation
to put full IMSRG(3) $\emax=6$ and hopefully $\emax=8$
within reach.
Additional restrictions can be placed on
for example the range of three-body channels,
potentially allowing IMSRG(3) calculations
to reach even larger single-particle model space sizes.
This will allow us to perform converged IMSRG(3) calculations
for medium-mass nuclei
and investigate what effects it has on observables.
Parallel to this,
obtaining exact (full configuration interaction) results
in $\emax=2$
would allow us to see how IMSRG(3)
compares with IMSRG(2)
relative to the exact solution
of the Schr{\"o}dinger equation
in $\emax=2$.
A secondary implementation goal
is the support of tensor operators,
which would give our implementation access to many more observables,
such as low-lying spectroscopy.

A key result of this thesis is that the organization by computational cost,
at least for the limited calcuations discussed here
for which we lack exact results
provided by full configuration interaction calculations,
is not a good strategy for developing an IMSRG(3) approximation.
To develop the IMSRG(3) further,
a more careful look into the behavior and structure of the fundamental commutators
is important
to understand the deficiencies in the naive organization by cost.
One path to doing this is the perturbative analysis
of the IMSRG(3) flow equations.
This could allow one to develop an approximate IMSRG(3) truncation
that is potentially fourth- or fifth-order exact in MBPT.
This analysis will need to be adapted
to account for non-HF reference states
and also cases beyond the NO2B approximation
where the residual normal-ordered three-body Hamiltonian
is included at the start of the IMSRG calculation.
In this analysis of the IMSRG(3),
it will be interesting to look at how the IMSRG(3)
performs relative to the IMSRG(2) under variation of the reference state.
If one finds that the IMSRG(3) is more successful
for sub-optimal reference states,
understanding this feature of the IMSRG(3)
may allow for constructions of IMSRG(3) approximations
that share this behavior.

This thesis represents a first step
towards full IMSRG(3) calculations
of light and medium-mass nuclei
and the systematic development and testing
of approximate IMSRG(3) truncation schemes.
Accessing full IMSRG(3) calculations
will provide access to high-precision many-body results,
hopefully bringing the many-body uncertainty error
for ground-state energies robustly into the per-mille range.
Well-motivated approximations of the IMSRG(3)
will make most of the accuracy benefits of the IMSRG(3)
available at a lower computational cost,
necessary for calculations where large model spaces
are required to converge calculations,
for example for heavy nuclei or hard, unevolved nuclear Hamiltonians.
