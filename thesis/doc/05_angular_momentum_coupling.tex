\chapter{Angular-momentum coupling for the IMSRG}\label{ch:ang_mom_coupling}

In theoretical physics,
the exploitation of symmetries is essential to making
the solution of certain problems computationally tractable.
In many-body theories,
general theories can be simplified
(in terms of computational cost)
by exploiting the symmetries
present in the system and the chosen single-particle basis.
For rotationally invariant systems,
this symmetry exploitation is called angular-momentum reduction (AMR),
which casts the many-body problem
into the language of spherical tensors
and angular-momentum eigenstates
and analytically simplifies
the angular-momentum-projection dependence
related to the geometry of spherical systems.
The angular-momentum reduction of a many-body approach
can reduce the storage and computational cost by orders of magnitude,
making it a very powerful tool to extend
the range of the approach to larger model spaces
or in some cases make calculations possible at all.
The IMSRG is an excellent target for angular-momentum reduction,
as one typically needs to push the model space
to quite large single-particle basis truncations
to achieve converged results.
In this chapter,
we give a brief overview of how angular-momentum reduction works
and discuss its application
to the IMSRG(3).

\section{Wigner-Eckart theorem}\label{sec:amr}

Core to the angular-momentum reduction formalism
are rotational symmetry,
formally described by the SU(2) Lie group,
and the generator of rotations,
the angular momentum $\vec{J}$.
The prerequisites for AMR are~\cite{Tich20jcoupling}:
a rotationally invariant Hamiltonian, which commutes with $\vec{J}$;
a spherical single-particle basis, which consists of eigenstates of $J^2$ and $J_{z}$;
and a spherical reference state,
which has good angular momentum $J=0$.
In this case, the Hamiltonian, the single-particle basis,
and the reference state share the symmetry group SU(2),
and one can cast the working equations of the theory
into a spherically symmetric form
and perform the analytical simplifications
to allow one to profit from AMR.

An operator $O$ can be expanded in spherical tensors $\mathbf{O}^{J}$ of rank $J$,
each with $2 J + 1$ components $O^{J}_{M}$~\cite{Suho07angmom}.
These spherical tensors have definite transformation behavior
under rotations.
Specifically, for a given unitary transformation $U(R)$
that corresponds to a rotation $R$,
the spherical tensor components transform like~\cite{Suho07angmom}
\begin{equation}
  U(R)O^{J}_{M}U^{\dagger}(R) = \sum_{M'=-J}^{J} D^{J}_{M'M}(R) O_{M'}^{J}\,,
\end{equation}
where $D^{J}_{M'M}(R)$ are the Wigner $D$ functions,
which also give the transformation of the spherical harmonics under the same rotation~\cite{Suho07angmom}:
\begin{equation}
  U(R)\ket{j m} = \sum_{m' = -j}^{j} D_{m' m}^{j}(R) \ket{j m'}.
\end{equation}
Spherical tensors can be further simplified by the Wigner-Eckart theorem.
The Wigner-Eckart theorem states
that the matrix elements of a spherical tensor
can be factorized into
a reduced matrix element that is operator-specific
and independent of any angular-momentum projection
(in the bra state, ket state, and the tensor component)
and a projection-dependent part
that contains only geometric information
and is independent of the specific operator~\cite{Wign27wet,Ecka30wet}:
\begin{equation}\label{eq:wet_theorem}
  \braket{\xi_1 j_1 m_1 | O^{J}_{M} | \xi_2 j_2 m_2 }
  = (-1 )^{2J} \frac{1}{\hat{\jmath}_1}
  \cgsymbol{j_2}{J}{j_1}{m_2}{M}{m_1}
  \braket{\xi_1 j_1 || \textbf{O}^{J} || \xi_2 j_2}\,,
\end{equation}
where $\hat{\jmath} \equiv \sqrt{2 j + 1}$.
The states $\ket{\xi j m}$
are eigenstates of
angular momentum squared $J^2$
and angular-momentum projection $J_{z}$,
with all other relevant quantum numbers
contained in $\xi$.
We generally use the shorthand $\ket{p} = \ket{\tilde{p} m_p}$,
where $\tilde{p} \equiv \xi_p j_p$, for simplicity.
Occasionally, we need time-reversed states
with flipped angular-momentum projections,
which we denote by $\ket{\bar{p}} \equiv \ket{\tilde{p} (-m_p)}$.
Note that the reduced single-particle index $\tilde{p}$ is the same
for normal states and time-reversed states.
Note that in Eq.~\eqref{eq:wet_theorem}
we have chosen the ``Wigner'' convention for reduced matrix elements,
which is also used by Suhonen, Edmonds, Racah, and Varshalovich
and which differs from the convention used by, for example, Sakurai.

Equipped with the factorization
provided by the Wigner-Eckart theorem
and the choice of spherical reference state
and single-particle basis,
one can in principle analytically perform the summations
over all the angular-momentum-projection quantum numbers
in a given expression.
Intuitively,
for every reduced index $\tilde{p}$ in a many-body expression,
one knows that both the reduced matrix elements
and the reference state do not depend on $m_p$,
so one is able to treat the entire ``shell''
of $2 j_p + 1$ states collectively.
This is where the ``reduction'' part of angular-momentum reduction
takes place,
since the resulting expressions
are completely independent of projection quantum numbers,
and one can use a reduced representation
for the remaining operator-specific information
that does not depend on the projection quantum numbers.

One particularly convenient approach to
doing angular-momentum reduction
is the diagrammatic expansion
of an expression in a so-called Jucys graph
and the simplification of the graph using various identities~\cite{Vars88angmom,Worm06angmom,Lind86angmom}.
This approach was automated in the \texttt{amc} code
published in Ref.~\cite{Tich20jcoupling}.
The \texttt{amc} program
automatically converts an uncoupled expression
provided to it via an input file written in the AMC language
into an equivalent $m$-independent reduced expression.
Accomplishing this reduction by hand
is tedious and error-prone,
so we used this program as the primary workhorse
for the angular-momentum reduction in Section~\ref{sec:amr_scalar}.

\section{Angular-momentum reduction for scalar operators}\label{sec:amr_scalar}

We will now apply angular-momentum reduction
to the in-medium similarity renormalization group.
The result of this symmetry reduction is the so-called $J$-scheme IMSRG,
where the flow equations are formulated
in terms of coupled or reduced matrix elements
and the expressions have been reduced
such that all dependence on angular-momentum projections
has been analytically simplified.
In the nuclear case,
scalar operators play a special role in the IMSRG
as the Hamiltonian and the generator are both scalar operators.
This means that the many-body formalisms
discussed in Chapters~\ref{ch:many_body} and~\ref{ch:imsrg}
can be symmetry reduced for closed-shell systems
by considering only the case of scalar spherical tensors.
As a result,
the $J$-scheme IMSRG can access energies and charge radii
without needing to consider the more complicated case
of general spherical tensor operators
(which we give an outlook on in Section~\ref{sec:tensor_jscheme}).

\subsection{Operator representation}

Since spherical scalars are invariant under rotations
and do not depend on any angular-momentum projection numbers,
one can work with coupled matrix elements
rather than reduced matrix elements
  [as in Eq.~\eqref{eq:wet_theorem}],
which differ by a simple factor
(shown here for a one-body operator):
\begin{equation}
  \braket{\tilde{p} | T^{0}_{0} | \tilde{q}} =
  \frac{1}{\hat{j_p}}\braket{\tilde{p} || \mathbf{T}^{0} || \tilde{q}}.
\end{equation}

For the IMSRG(3), we need coupled matrix elements
for up to three-body operators.
Thus, we need to couple our $A$-body basis
to be made up of eigenstates of
the $A$-body total angular momentum squared $J^2$
and the $z$-component of the total angular momentum $J_{z}$.
Since our chosen single-particle basis $\ket{p} = \ket{\tilde{p}m_p}$
consists of eigenstates of $J_{\text{1B}}^2$ and $J_{\text{1B},z}$
for the one-body angular momentum $\vec{J}_{\text{1B}}$,
the coupled matrix elements of a scalar one-body operator
are simply the uncoupled matrix elements
\footnote{
  In Eq.~\eqref{eq:onebody_coupled_to_uncoupled},
  $p$ and $q$ have their angular-momentum projections
  implicitly fixed to $m_p=m_q=1/2$.
},
\begin{equation}
  O_{\tilde{p} \tilde{q}} = \braket{\tilde{p} m_p=1/2 | O | \tilde{q} m_q=1/2} \delta_{j_p j_q}
  = O_{pq}\,, \label{eq:onebody_coupled_to_uncoupled}
\end{equation}
where we have explicitly denoted that $O$ is diagonal in $j_p$
\footnote{We previously denoted $A$-body operators by $\abodyop{O}$.
  From this point on, we will leave off the (redundant) superscript
  to reduce notational clutter,
  as the many-body rank of an operator
  can be inferred from the number of indices on its matrix elements.}.
Note that $m_p = m_q$ could be any other value
within the range given by $j_p$;
the matrix elements of a scalar operator will not change due to a different choice.
We used $m_p=1/2$ here simply because it will always be a valid choice
for all $j_p$ in our single-particle basis.

The representation of one-body operators can be made a bit more clear
by adding an angular-momentum ``channel'' to the notation:
\begin{equation}
  O^{j}_{pq}\,.
\end{equation}
The channel $j$ indicates that the matrix elements of $O_{pq}$
are diagonal in $j_p=j_q$,
and in this channel only the indices $p$ and $q$ with
$j_p = j_q = j$ have non-zero matrix elements.
We will stick with this notation
for coupled one-body matrix elements going forward.

Our antisymmetrized two-body states
\begin{equation}
  \ket{pq} = \crea{p} \crea{q} \ket{0}
\end{equation}
are not eigenstates of $J_{\text{2B}}^2$,
only of $J_{\text{2B},z}$.
Coupling the states to two-body angular momentum $J_{pq}$
gives the coupled two-body basis,
\begin{equation}\label{eq:unnormalized_twobody_basis}
  \ket{(\tilde{p} \tilde{q}) J_{pq} M_{pq}} \equiv
  \sum_{m_p, m_q}
  \cgsymbol{j_p}{j_q}{J_{pq}}{m_p}{m_q}{M_{pq}}
  \ket{pq}
  \,,
\end{equation}
where the coupling brackets in $\ket{(\tilde{p} \tilde{q}) J_{pq} M_{pq}}$
indicate that $j_p$ and $j_q$ are coupled to $J_{pq}$
(and the rest of the quantum numbers in $\tilde{p}$ and $\tilde{q}$ are not involved).
Note that the Clebsch-Gordan coefficients
\begin{equation}
  \cgsymbol{j_p}{j_q}{J_{pq}}{m_p}{m_q}{M_{pq}}
\end{equation}
are defined such that they are 0
if $M_{pq} \neq m_p + m_q$,
collapsing one of the sums over angular-momentum projections.
The two-body states in Eq.~\eqref{eq:unnormalized_twobody_basis}
are not normalized,
which can be remedied by multiplying them with the following factor~\cite{Suho07angmom}:
\begin{equation}
  N_{pq(J_{pq})} \equiv \frac{\sqrt{1 + (-1)^{J_{pq}}
      \delta_{\tilde{p} \tilde{q}}}}{1 + \delta_{\tilde{p} \tilde{q}}}.
\end{equation}
However, the coupled many-body expressions
(and their numerical implementations)
are simpler when using unnormalized coupled two- and three-body matrix elements.

The unnormalized coupled two-body matrix elements of
a scalar two-body operator $O$ are given by
\begin{samepage}
  \begin{align}
    O^{J_{pq}}_{\tilde{p}\tilde{q}\tilde{r}\tilde{s}}
     & \equiv \braket{
      (\tilde{p} \tilde{q})J_{pq} M_{pq}=0|
      O
      |(\tilde{r} \tilde{s})J_{pq} M_{pq}=0
    }                           \\
     & =\sum_{\substack{m_p,m_q \\ m_r,m_s}}
    \cgsymbol{j_p}{j_q}{J_{pq}}{m_p}{m_q}{M_{pq} = 0}
    \cgsymbol{j_r}{j_s}{J_{pq}}{m_r}{m_s}{M_{pq} = 0}
    O_{pqrs}\,.\label{eq:unnormalized_twobody_mels}
  \end{align}
\end{samepage}
Our representation builds the diagonality of the matrix elements
in $J_{pq} = J_{rs}$ into the notation,
as indicated by the $J_{pq}$ channel in the superscript.
Again, $M_{pq} = 0$ is a choice we have made that works
for all sets of $\tilde{p}$ and $\tilde{q}$.
It is worth noting that not all $\tilde{p} \tilde{q}$
(or $\tilde{r} \tilde{s}$) combinations
can couple to a given $J_{pq}$.
This is not a problem, because in Eq.~\eqref{eq:unnormalized_twobody_mels}
the Clebsch-Gordan coefficients cause those matrix elements to be 0.
This means they do not unphysically contribute in any many-body expressions.
In numerical implementations,
the exploitation of this reduction in the number of valid two-body states
in a given angular-momentum channel
is essential to improving performance to the point
where calculations in large model spaces are possible.

The definition of the coupled three-body basis follows analogously,
\begin{equation}\label{eq:unnormalized_threebody_basis}
  \ket{[(\tilde{p} \tilde{q}) J_{pq} \tilde{r}] J_{pqr} M_{pqr}} \equiv
  \sum_{m_r, M_{pq}}
  \cgsymbol{J_{pq}}{j_r}{J_{pqr}}{M_{pq}}{m_r}{M_{pqr}}
  \sum_{m_p, m_q}
  \cgsymbol{j_p}{j_q}{J_{pq}}{m_p}{m_q}{M_{pq}}
  \ket{pqr}
  \,,
\end{equation}
where we have selected the ``standard'' coupling order
with $j_p$ and $j_q$ coupled first to $J_{pq}$
and then $J_{pq}$ and $j_r$ coupled to $J_{pqr}$.
Once again, these states are not normalized,
but working with unnormalized states is more convenient.

The unnormalized coupled matrix elements of a three-body operator $O$
are given by
\begin{align}
  \phantom{O^{(J_{pqr}, J_{pq}, J_{st})}_{\tilde{p}\tilde{q}\tilde{r}\tilde{s}\tilde{t}\tilde{u}}}
   & \begin{aligned}
    \mathllap{O^{(J_{pqr}, J_{pq}, J_{st})}_{\tilde{p}\tilde{q}\tilde{r}\tilde{s}\tilde{t}\tilde{u}}}
    \equiv \braket{
    [(\tilde{p} \tilde{q})J_{pq} \tilde{r}] J_{pqr} M_{pqr} = 1/2 |
    O
    | [(\tilde{s} \tilde{t})J_{st} \tilde{u}] J_{pqr} M_{pqr} = 1/2
    }
  \end{aligned} \\
   & \begin{aligned}
    \mathllap{}
    = \sum_{\substack{m_p,m_q,m_r,M_{pq}                     \\ m_s,m_t,m_u,M_{st}}} &
    \cgsymbol{j_p}{j_q}{J_{pq}}{m_p}{m_q}{M_{pq}}
    \cgsymbol{j_s}{j_t}{J_{st}}{m_s}{m_t}{M_{st}}            \\
     & \cgsymbol{J_{pq}}{j_r}{J_{pqr}}{M_{pq}}{m_r}{M_{pqr}=1/2}
    \cgsymbol{J_{st}}{j_u}{J_{pqr}}{M_{st}}{m_u}{M_{pqr}=1/2}    \\
     & O_{pqrstu}
    \,.
  \end{aligned}
\end{align}
The channel structure of the three-body matrix elements
has grown more complicated,
with the channel including the diagonal total three-body angular momentum
$J_{pqr} = J_{stu}$
and the intermediate couplings $J_{pq}$ and $J_{st}$.
These intermediate couplings can take on different values,
substantially increasing the number of three-body channels
for which we need to handle matrix elements.
Again, $M_{pqr} = 1/2$ is a choice we have made that works
for all sets of $\tilde{p}$, $\tilde{q}$, and $\tilde{r}$.

At this point, it is useful to discuss the symmetry properties
of these matrix elements.
For a Hermitian operator
the coupled matrix elements have the following properties:
\begin{samepage}
  \begin{subequations}
    \begin{align}
      O_{\tilde{p}\tilde{q}}                                                                 & = O_{\tilde{q}\tilde{p}}\,, \\
      O^{J_{pq}}_{\tilde{p}\tilde{q}\tilde{r}\tilde{s}}                                      & =
      O^{J_{pq}}_{\tilde{r}\tilde{s}\tilde{p}\tilde{q}}\,,                                                                 \\
      O^{(J_{pqr}, J_{pq}, J_{st})}_{\tilde{p}\tilde{q}\tilde{r}\tilde{s}\tilde{t}\tilde{u}} & =
      O^{(J_{pqr}, J_{st}, J_{pq})}_{\tilde{s}\tilde{t}\tilde{u}\tilde{p}\tilde{q}\tilde{r}}\,.\label{eq:coupled_herm_threebody}
    \end{align}
  \end{subequations}
\end{samepage}
Note the transposition of the intermediate couplings
in the three-body channel in Eq.~\eqref{eq:coupled_herm_threebody}.
For an anti-Hermitian operator
the coupled matrix elements have the following properties:
\begin{subequations}
  \begin{align}
    O_{\tilde{p}\tilde{q}}                                                                 & = - O_{\tilde{q}\tilde{p}}\,, \\
    O^{J_{pq}}_{\tilde{p}\tilde{q}\tilde{r}\tilde{s}}                                      & =
    - O^{J_{pq}}_{\tilde{r}\tilde{s}\tilde{p}\tilde{q}}\,,                                                                 \\
    O^{(J_{pqr}, J_{pq}, J_{st})}_{\tilde{p}\tilde{q}\tilde{r}\tilde{s}\tilde{t}\tilde{u}} & =
    - O^{(J_{pqr}, J_{st}, J_{pq})}_{\tilde{s}\tilde{t}\tilde{u}\tilde{p}\tilde{q}\tilde{r}}\,.\label{eq:coupled_antiherm_threebody}
  \end{align}
\end{subequations}

Our two- and three-body matrix elements are also antisymmetric,
although this symmetry is not realized as simply
for coupled matrix elements as it is for uncoupled matrix elements.
Two-body matrix elements have the following properties:
\begin{subequations}
  \begin{align}
    O_{\tilde{p}\tilde{q}\tilde{r}\tilde{s}}^{J_{pq}} & =
    - (-1)^{j_p + j_q - J_{pq}}
    O_{\tilde{q}\tilde{p}\tilde{r}\tilde{s}}^{J_{pq}}\,,  \\
    O_{\tilde{p}\tilde{q}\tilde{r}\tilde{s}}^{J_{pq}} & =
    - (-1)^{j_r + j_s - J_{pq}}
    O_{\tilde{p}\tilde{q}\tilde{s}\tilde{r}}^{J_{pq}}\,.
  \end{align}
\end{subequations}
Three-body matrix elements have the following properties:
\begin{subequations}
  \begin{align}
    O_{\tilde{p}\tilde{q}\tilde{r}\tilde{s}\tilde{t}\tilde{u}}^{(J_{pqr}, J_{pq}, J_{st})} & =
    - (-1)^{j_p + j_q - J_{pq}}
    O_{\tilde{q}\tilde{p}\tilde{r}\tilde{s}\tilde{t}\tilde{u}}^{(J_{pqr}, J_{pq}, J_{st})}\,,  \\
    O_{\tilde{p}\tilde{q}\tilde{r}\tilde{s}\tilde{t}\tilde{u}}^{(J_{pqr}, J_{pq}, J_{st})} & =
    \hat{J}_{pq}
    \sum_{J_{2}} \hat{J}_{2}
    \sixj{j_p}{j_q}{J_{pq}}{j_r}{J_{pqr}}{J_{2}}
    O_{\tilde{r}\tilde{q}\tilde{p}\tilde{s}\tilde{t}\tilde{u}}^{(J_{pqr}, J_{2}, J_{st})}\,,   \\
    O_{\tilde{p}\tilde{q}\tilde{r}\tilde{s}\tilde{t}\tilde{u}}^{(J_{pqr}, J_{pq}, J_{st})} & =
    - (-1)^{j_q + j_r - J_{pq}}
    \hat{J}_{pq}
    \sum_{J_{2}} \hat{J}_{2} (-1)^{J_{2}}
    \sixj{j_q}{j_p}{J_{pq}}{j_r}{J_{pqr}}{J_{2}}
    O_{\tilde{p}\tilde{r}\tilde{q}\tilde{s}\tilde{t}\tilde{u}}^{(J_{pqr}, J_{2}, J_{st})}\,,   \\
    O_{\tilde{p}\tilde{q}\tilde{r}\tilde{s}\tilde{t}\tilde{u}}^{(J_{pqr}, J_{pq}, J_{st})} & =
    - (-1)^{j_s + j_t - J_{st}}
    O_{\tilde{p}\tilde{q}\tilde{r}\tilde{t}\tilde{s}\tilde{u}}^{(J_{pqr}, J_{pq}, J_{st})}\,,  \\
    O_{\tilde{p}\tilde{q}\tilde{r}\tilde{s}\tilde{t}\tilde{u}}^{(J_{pqr}, J_{pq}, J_{st})} & =
    \hat{J}_{st}
    \sum_{J_{2}} \hat{J}_{2}
    \sixj{j_s}{j_t}{J_{st}}{j_u}{J_{pqr}}{J_{2}}
    O_{\tilde{p}\tilde{q}\tilde{r}\tilde{u}\tilde{t}\tilde{s}}^{(J_{pqr}, J_{pq}, J_{2})}\,,   \\
    O_{\tilde{p}\tilde{q}\tilde{r}\tilde{s}\tilde{t}\tilde{u}}^{(J_{pqr}, J_{pq}, J_{st})} & =
    - (-1)^{j_t + j_u - J_{st}}
    \hat{J}_{st}
    \sum_{J_{2}} \hat{J}_{2} (-1)^{J_{2}}
    \sixj{j_t}{j_s}{J_{st}}{j_u}{J_{pqr}}{J_{2}}
    O_{\tilde{p}\tilde{q}\tilde{r}\tilde{s}\tilde{u}\tilde{t}}^{(J_{pqr}, J_{pq}, J_{2})}\,.
  \end{align}
\end{subequations}
We provide only the pairwise antisymmetry properties,
as the relations for the remaining permutations
can be obtained by applying pairwise permutations.

At this point, we drop the cumbersome ``tilde'' notation
for reduced state indices.
The channels on matrix elements should make it clear
whether the matrix elements are coupled or not.
Additionally, in the surrounding text,
we make it clear whether an expression
is coupled or uncoupled.

\subsection{Coupled many-body expressions}\label{sec:jscheme_many_body_expressions}

In addition to the fundamental commutators,
IMSRG calculations typically employ a couple standard operations,
which we discuss here.
The first is bringing the initial Hamiltonian into normal order
with respect to the employed $A$-body reference state.
Given a Hamiltonian $H$ with one- through three-body parts,
the coupled expressions for
the matrix elements of the normal-ordered Hamiltonian are
\begin{subequations}
  \begin{align}
    \phantom{W_{pqrstu}^{(J_{pqr}, J_{pq}, J_{st})}}
     & \begin{aligned}
      \mathllap{E} = \bar{H} & = \sum_{j_{a}} \hat{\jmath}^{2}_a \sum_{a} H_{aa}^{j_{a}}
      + \frac{1}{2} \sum_{J_{ab}} \hat{J}_{ab}^2 \sum_{ab} H_{abab}^{J_{ab}}             \\
                             & \quad
      + \frac{1}{6} \sum_{J_{ab} J_{abc}} \hat{J}_{abc}^2 \sum_{abc}  H_{abcabc}^{(J_{abc}, J_{ab}, J_{ab})}\,,
    \end{aligned} \\
     & \begin{aligned}
      \mathllap{f_{pq}^{j_p}} = \bar{H}_{pq}^{j_p} &
      = H_{pq}^{j_p}
      + \frac{1}{\hat{\jmath}_{p}^2} \sum_{J_{pa}} \hat{J}_{pa}^2 \sum_{a} H_{paqa}^{J_{pa}} \\
                                                   & \quad
      + \frac{1}{2 \hat{\jmath}_{p}^2} \sum_{J_{pa} J_{pab}} \hat{J}_{pab}^2 \sum_{ab}  H_{pabqab}^{(J_{pab}, J_{pa}, J_{pa})}\,,
    \end{aligned} \\
     & \begin{aligned}
      \mathllap{\Gamma_{pqrs}^{J_{pq}}} = \bar{H}_{pqrs}^{J_{pq}}
       & = H_{pqrs}^{J_{pq}}
      + \frac{1}{\hat{J}_{pq}^2} \sum_{J_{pqa}} \hat{J}_{pqa}^2 \sum_{a} H_{pqarsa}^{(J_{pqa}, J_{pq}, J_{pq})} \,,
    \end{aligned} \\
     & \begin{aligned}
      \mathllap{W_{pqrstu}^{(J_{pqr}, J_{pq}, J_{st})}} = \bar{H}_{pqrstu}^{(J_{pqr}, J_{pq}, J_{st})} & = H_{pqrstu}^{(J_{pqr}, J_{pq}, J_{st})} \,.
    \end{aligned}
  \end{align}
\end{subequations}
Recall our previous convention that
the indices $p$, $q$, $r$, \ldots\ run over all single-particle states,
the indices $i$, $j$, $k$, \ldots\ run over hole states,
and the indices $a$, $b$, $c$, \ldots\ run over particle states.

Next, we need an antisymmetrizer for our two- and three-body matrix elements.
This is because the fundamental commutators in Section~\ref{sec:jscheme_fundamental_comm}
are not antisymmetrized to simplify the expressions and the numerical implementations.
Thus, we need to explicitly restore the antisymmetry of some index combinations
after evaluating the non-antisymmetrized fundamental commutators.
In the two-body case,
the antisymmetrizer is
\begin{equation}\label{eq:twobody_antisymmetrizer}
  \mathcal{A}_{\text{2B}} \equiv\mathcal{A}_{pq} \mathcal{A}_{rs}\,,
\end{equation}
with
\begin{equation}
  \mathcal{A}_{pq} \equiv \frac{1}{2}(1 - P_{pq})\,.
\end{equation}
Recall that in the uncoupled expressions in Chapter~\ref{ch:imsrg}
the permutation operator $P_{pq}$ simply exchanged the indices $p$ and $q$ in the following expression.
Now that indices are coupled,
the action of permutation operators is complicated
by the fact that they also change the coupling order.
To implement $\mathcal{A}_{\text{2B}}$,
all we need are the expressions for the action of $P_{pq}$
and $P_{rs}$ on two-body matrix elements:
\begin{subequations}
  \begin{align}
    O_{pqrs}^{J_{pq}} & \xleftarrow{P_{pq}} (-1)^{j_p + j_q - J_{pq}} O_{qprs}^{J_{pq}}\,, \\
    O_{pqrs}^{J_{pq}} & \xleftarrow{P_{rs}} (-1)^{j_r + j_s - J_{pq}} O_{pqsr}^{J_{pq}}\,.
  \end{align}
\end{subequations}
Using these operations along with scalar multiplication of and addition of matrix elements,
the implementation of a two-body antisymmetrizer is simple.

In the three-body case,
the antisymmetrizer is
\begin{equation}\label{eq:threebody_antisymmetrizer}
  \mathcal{A}_{\text{3B}} \equiv \mathcal{A}_{pqr} \mathcal{A}_{stu}\,,
\end{equation}
with
\begin{equation}
  \mathcal{A}_{pqr} \equiv \frac{1}{6}(1 + P_{prq} + P_{prq}^2)(1 - P_{pq})\,.
\end{equation}
Here $P_{prq}$ cyclically permutes the indices $p$, $q$, and $r$ such that
\begin{equation}
  (p, q, r) \xrightarrow{P_{prq}} (q, r, p) \xrightarrow{P_{prq}} (r, p, q) \xrightarrow{P_{prq}} (p, q, r)\,.
\end{equation}
There are other ways to define $\mathcal{A}_{pqr}$,
for example in terms of $P_{pq}$ and $P_{qr}$.
They are all equivalent,
and which one one chooses is a matter of preference.
The action of $P_{prq}$, $P_{pq}$, $P_{sut}$, and $P_{st}$
on three-body matrix elements is given by
\begin{subequations}
  \begin{align}
    O_{pqrstu}^{(J_{pqr}, J_{pq}, J_{st})} &
    \xleftarrow{P_{prq}}
    -1 (-1)^{j_p + j_q - J_{pq}} \hat{J}_{pq}
    \sum_{J_{2}} \hat{J}_{2}
    \sixj{j_q}{j_p}{J_{pq}}{j_r}{J_{pqr}}{J_{2}}
    O_{rpqstu}^{(J_{pqr}, J_{2}, J_{st})}\,,  \\
    O_{pqrstu}^{(J_{pqr}, J_{pq}, J_{st})} &
    \xleftarrow{P_{pq}}
    (-1)^{j_p + j_q - J_{pq}}
    O_{qprstu}^{(J_{pqr}, J_{pq}, J_{st})}\,, \\
    O_{pqrstu}^{(J_{pqr}, J_{pq}, J_{st})} &
    \xleftarrow{P_{sut}}
    -1 (-1)^{j_s + j_t - J_{st}} \hat{J}_{st}
    \sum_{J_{2}} \hat{J}_{2}
    \sixj{j_t}{j_s}{J_{st}}{j_u}{J_{pqr}}{J_{2}}
    O_{pqrust}^{(J_{pqr}, J_{pq}, J_{2})}\,,  \\
    O_{pqrstu}^{(J_{pqr}, J_{pq}, J_{st})} &
    \xleftarrow{P_{st}}
    (-1)^{j_s + j_t - J_{st}}
    O_{pqrtsu}^{(J_{pqr}, J_{pq}, J_{st})}\,.
  \end{align}
\end{subequations}
With these basic operations in hand,
one can implement a general three-body antisymmetrizer.

Finally, the second-order M{\o}ller-Plesset MBPT (MP2) energy correction
is a critical diagnostic for the IMSRG
as it is directly proportional to the matrix elements
that the IMSRG evolution should suppress.
The coupled expression for the MP2 energy correction is
\begin{equation}
  E_{\text{MP2}} =
  - \sum_{j_{a}} \hat{\jmath}_{a}^2 \sum_{ai} \frac{|f_{ai}^{j_a}|^2}{\epsilon_{i}^{a}}
  - \frac{1}{4} \sum_{J_{ab}} \hat{J}_{ab}^2 \sum_{abij}
  \frac{|\Gamma_{abij}^{J_{ab}}|^2}{\epsilon_{ij}^{ab}}
  - \frac{1}{36} \sum_{J_{abc} J_{ab} J_{ij}} \hat{J}_{abc}^2 \sum_{abcijk}
  \frac{|W_{abcijk}^{(J_{abc}, J_{ab}, J_{ij})}|^2}{\epsilon_{ijk}^{abc}}\,,
\end{equation}
with
\begin{align}
  \epsilon_{ij\cdots}^{ab\cdots} & = e_a + e_b + \cdots - (e_i + e_j + \cdots)\,, \\
  e_p                            & = f_{pp}\,,
\end{align}
as in Eq.~\eqref{eq:mp_energy_denom}.

\subsection{Coupled fundamental commutators}\label{sec:jscheme_fundamental_comm}

In this section,
we present the coupled expressions for
the fundamental commutators of two scalar operators.
These expressions were obtained using
the \texttt{amc} code from Ref.~\cite{Tich20jcoupling}
on the uncoupled commutator expressions
given in Appendix~\ref{app:mscheme_fundamental_commutators}.
While this approach immediately produced desirable results
for some commutators,
for many it was possible to simplify the resulting expressions
by exploiting symmetries of the uncoupled expressions.
We discuss examples of these simplifications in Appendix~\ref{app:jscheme_commutator_tricks}.
Here, we will only show the simplest expressions
we were able to produce.

The commutator expressions are ``fundamental'' in the sense
that they are the basic operations required for any IMSRG(3) implementation.
Using these expressions,
one can quickly combine and expand them to produce
the full IMSRG(3) flow equations (see Section~\ref{sec:imsrgthree}),
and one can also easily implement
the series of nested commutators required by
the Magnus and BCH expansions (see Section~\ref{sec:imsrg_magnus}).
Thus, correctly and efficiently implementing these expressions
constitutes the main challenge of any IMSRG implementation.

The commutator of a $K$-body and an $L$-body operator
gives an operator with \mbox{$|K-L|$-} to $(K+L-1)$-body parts:
\begin{equation}
  \left[A^{(K)}, B^{(L)}\right] = \sum_{M=|K - L|}^{K + L - 1} C^{(M)}\,.
\end{equation}
The expressions we give below are
for the coupled matrix elements of the different $M$-body terms.
In the IMSRG(3), we discard any four- and five-body parts induced,
which appear in principle
in the commutator of a two-body and a three-body operator
and in the commutator of two three-body operators.
We focus on the case where $A$ and $B$ and thus the resulting $C$ are scalar operators.

Note that expressions for two- and three-body matrix elements
are not antisymmetrized,
so the appropriate antisymmetrizer must be applied
  [see Eqs.~\eqref{eq:twobody_antisymmetrizer} and~\eqref{eq:threebody_antisymmetrizer}]
to the matrix elements after evaluating the commutator.
We break our typical index label convention
to use the convention that the index labels $i$,~$j$,~$k$,~\ldots\
are reserved for external indices,
that is, indices appearing on the matrix elements of the resulting operator,
and the index labels $a$,~$b$,~$c$,~\ldots\
are reserved for contracted indices,
that is, indices that are summed over in the matrix elements of the input operators.
As in Chapter~\ref{ch:imsrg}, $\bar{n}_a = 1 - n_a$.

\subsubsection{
  \texorpdfstring{$[\onebodyop{A}, \onebodyop{B}]$}{[1, 1]}
}

The commutator of two one-body operators has
zero- and one-body parts.

The resulting zero-body contribution is given by the coupled expression
\begin{equation}
  C = \sum_{j_a} \hat{\jmath}^{2}_{a} \sum_{ab} (n_a - n_b) A_{ab}^{j_a} B_{ba}^{j_a}\,.
\end{equation}

The resulting coupled one-body matrix elements are given by the coupled expression
\begin{equation}
  C_{ij}^{j_i} = \sum_{a}(A_{ia}^{j_i} B_{aj}^{j_i} - B_{ia}^{j_i} A_{aj}^{j_i})\,.
\end{equation}

\subsubsection{
  \texorpdfstring{$[\onebodyop{A}, \twobodyop{B}]$}{[1, 2]}
}

The commutator of a one-body operator and a two-body operator has
one- and two-body parts.

The resulting coupled one-body matrix elements are given by the coupled expression
\begin{equation}
  C_{ij}^{j_i} = \frac{1}{\hat{\jmath}_{i}^2} \sum_{j_a} \sum_{J_2} \hat{J}_{2}^2
  \sum_{ab} (n_a \bar{n}_b - \bar{n}_a n_b) A_{ab}^{j_a} B_{iajb}^{J_2}\,.
\end{equation}

The resulting coupled two-body matrix elements are given by the coupled expression
\begin{equation}
  C_{ijkl}^{J_C} = 2 \sum_{j_a} \sum_{a} \left(
  A_{ia}^{j_a} B_{ajkl}^{J_C} - A_{ak}^{j_a} B_{ijal}^{J_C}
  \right)
  \,.
\end{equation}

\subsubsection{
  \texorpdfstring{$[\twobodyop{A}, \twobodyop{B}]$}{[2, 2]}
}

The commutator of two two-body operators has
zero- through three-body parts.

The resulting zero-body contribution is given by the coupled expression
\begin{equation}
  C = \frac{1}{4}\sum_{J_{ab}} \hat{J}_{ab}^{2}\sum_{abcd}
  (n_a n_b \bar{n}_c \bar{n}_d - \bar{n}_a \bar{n}_b n_c n_d)
  A_{abcd}^{J_{ab}} B_{cdab}^{J_{ab}}\,.
\end{equation}

The resulting coupled one-body matrix elements are given by the coupled expression
\begin{equation}
  C_{ij}^{j_i} = \frac{1}{2} \frac{1}{\hat{\jmath}_{i}^2}
  \sum_{J_{ab}} \hat{J}_{ab}^2
  \sum_{abc}
  (\bar{n}_a \bar{n}_b n_c + n_a n_b \bar{n}_c)
  (A_{ciab}^{J_{ab}} B_{abcj}^{J_{ab}} - B_{ciab}^{J_{ab}} A_{abcj}^{J_{ab}})\,.
\end{equation}

The resulting coupled two-body matrix elements are given by the coupled expression
\begin{equation}
  C^{J_C}_{ijkl}
  =
  D^{J_C}_{ijkl}
  + E^{J_C}_{ijkl}\,,
\end{equation}
with
\begin{align}
  D^{J_C}_{ijkl}                        & \equiv \frac{1}{2} \sum_{ab}
  (\bar{n}_a \bar{n}_b - n_a n_b)
  \left(
  A_{ijab}^{J_C} B_{abkl}^{J_C} - B_{ijab}^{J_C} A_{abkl}^{J_C}
  \right)\,,                                                           \\
  \overline{E}_{i\bar{l}k\bar{j}}^{J_C} & \equiv 4 \sum_{ab}
  (n_a \bar{n}_b - \bar{n}_a n_b)
  \overline{A}_{a\bar{b}k\bar{j}}^{J_C}
  \overline{B}_{i\bar{l}a\bar{b}}^{J_C}\,. \label{eq:comm222_term2}
\end{align}
The expression in Eq.~\eqref{eq:comm222_term2} is written
in terms of Pandya-transformed matrix elements,
where the two-body Pandya transformation is given by~\cite{Pand56pandya}
\begin{equation}
  \overline{O}_{p\bar{q}r\bar{s}}^{J_{2}} \equiv
  - \sum_{J_{2}'}
  \hat{J}_{2}^{\prime 2}
  \sixj{j_p}{j_q}{J_{2}}{j_r}{j_s}{J_{2}'}
  O_{prqs}^{J_{2}'}\,.
\end{equation}
The indices $\bar{q}$ and $\bar{s}$ are time reversed,
that is, their angular-momentum projections are flipped.
Since we are working with reduced indices
without angular-momentum projections everywhere,
this distinction is irrelevant,
but we keep the modified indices around to be formally precise.
The Pandya-transformed matrix elements
are useful intermediates that isolate recoupling on $A$, $B$, and $E$
that can be done independently.
The resulting $\overline{E}$ must be
converted back to normal coupled matrix elements
by a final Pandya transformation
before being added to $D$ to give $C$.

The resulting coupled three-body matrix elements are given by the coupled expression
\begin{equation}
  \begin{split}
    C_{ijklmn}^{(J_{C,3}, J_{C,ij}, J_{C,lm})} & =
    -9 (-1)^{J_{C,lm}} \hat{J}_{C,ij} \hat{J}_{C,lm}
    \sum_{a} (-1)^{j_a + j_k}
    \sixj{j_k}{j_a}{J_{C,lm}}{j_{n}}{J_{C,3}}{J_{C,ij}} \\
    & \quad \quad \times \left(
    A_{ijna}^{J_{C,ij}} B_{aklm}^{J_{C,lm}}
    - B_{ijna}^{J_{C,ij}} A_{aklm}^{J_{C,lm}}
    \right).
  \end{split}
\end{equation}

\subsubsection{
  \texorpdfstring{$[\onebodyop{A}, \threebodyop{B}]$}{[1, 3]}
}

The commutator of a one-body operator and a three-body operator has
two- and three-body parts.

The resulting coupled two-body matrix elements are given by the coupled expression
\begin{equation}
  C_{ijkl}^{J_C} = \frac{1}{\hat{J}_{C}^2}
  \sum_{j_a}
  \sum_{J_{B,3}} \hat{J}_{B,3}^2
  \sum_{ab} (n_a \bar{n}_b - \bar{n}_a n_b)
  A_{ab}^{j_a} B_{ijbkla}^{(J_{B,3}, J_{C}, J_{C})}\,.
\end{equation}

The resulting coupled three-body matrix elements are given by the coupled expression
\begin{equation}
  C_{ijklmn}^{(J_{C,3}, J_{C,ij}, J_{C,lm})} =
  3 \sum_{j_a} \sum_{a}
  \left(
  A_{ia}^{j_a} B_{ajklmn}^{(J_{C,3}, J_{C,ij}, J_{C,lm})}
  - A_{la}^{j_a} B_{ijkamn}^{(J_{C,3}, J_{C,ij}, J_{C,lm})}
  \right).
\end{equation}

\subsubsection{
  \texorpdfstring{$[\twobodyop{A}, \threebodyop{B}]$}{[2, 3]}
}

The commutator of a two-body operator and a three-body operator has
one- through three-body parts.

The resulting coupled one-body matrix elements are given by the coupled expression
\begin{equation}
  \begin{split}
    C_{ij}^{j_i} &= - \frac{1}{4} \frac{1}{\hat{\jmath}_{i}^2}
    \sum_{(J_{B,3}, J_{ab})}
    \hat{J}_{B,3}^2
    \sum_{abcd}
    (n_a n_b \bar{n}_c \bar{n}_d - \bar{n}_a \bar{n}_b n_c n_d)
    A_{cdab}^{J_{ab}} B_{abicdj}^{(J_{B,3}, J_{ab}, J_{ab})}\,.
  \end{split}
\end{equation}

The resulting coupled two-body matrix elements are given by the coupled expression
\begin{equation}
  \begin{split}
    C_{ijkl}^{J_C} &= \frac{(-1)^{J_C}}{\hat{J}_C}
    \sum_{J_A, J_{B,3}} \hat{J}_A \hat{J}_{B,3}^2
    \sum_{abc}
    (n_a \bar{n}_b \bar{n}_c + \bar{n}_a n_b n_c) \\
    & \quad \left(
    (-1)^{j_k + j_l}
    \sixj{j_l}{j_k}{J_C}{j_a}{J_{B,3}}{J_A}
    A_{bcak}^{J_A} B_{ijabcl}^{(J_{B,3}, J_C, J_A)}
    \right.\\
    &\quad\quad \left. + (-1)^{j_i + j_j}
    \sixj{j_j}{j_i}{J_C}{j_a}{J_{B,3}}{J_A}
    A_{bcai}^{J_A} B_{klabcj}^{(J_{B,3}, J_C, J_A)}
    \right)\,.
  \end{split}
\end{equation}

The resulting coupled three-body matrix elements are given by the coupled expression
\begin{equation}
  C_{ijklmn}^{(J_{C,3}, J_{C, ij}, J_{C, lm})}
  = D_{ijklmn}^{(J_{C,3}, J_{C, ij}, J_{C, lm})}
  + E_{ijklmn}^{(J_{C,3}, J_{C, ij}, J_{C, lm})}\,,
\end{equation}
with
\begin{align}
  D_{ijklmn}^{(J_{C,3}, J_{C, ij}, J_{C, lm})}
   & \equiv \frac{3}{2} \sum_{ab} (\bar{n}_a \bar{n}_b - n_a n_b)
  A_{ijab}^{J_{C, ij}}
  B_{abklmn}^{(J_{C,3}, J_{C, ij}, J_{C, lm})}\,,                 \\
  E_{ijklmn}^{(J_{C,3}, J_{C,ij}, J_{C,lm})}
   & \equiv
  -\frac{3}{2} \sum_{ab} (\bar{n}_a \bar{n}_b - n_a n_b)
  A_{ablm}^{J_{C, lm}}
  B_{ijkabn}^{(J_{C,3}, J_{C, ij}, J_{C, lm})}\,.
\end{align}

\subsubsection{
  \texorpdfstring{$[\threebodyop{A}, \threebodyop{B}]$}{[3, 3]}
}

The commutator of two three-body operators has
zero- through three-body parts.

The resulting zero-body contribution is given by the coupled expression
\begin{equation}
  C = \frac{1}{36}
  \sum_{(J_3, J_{ab}, J_{de})} \hat{J}_{3}^2
  \sum_{abcdef}
  (
  n_a n_b n_c \bar{n}_d \bar{n}_e \bar{n}_f
  - \bar{n}_a \bar{n}_b \bar{n}_c n_d n_e n_f
  )
  A_{abcdef}^{(J_{3}, J_{ab}, J_{de})}
  B_{defabc}^{(J_{3}, J_{de}, J_{ab})}\,.
\end{equation}

The resulting coupled one-body matrix elements are given by the coupled expression
\begin{equation}
  \begin{split}
    C_{ij}^{j_i} & = \frac{1}{12} \frac{1}{\hat{j}_{i}^2}
    \sum_{(J_{3}, J_{ab}, J_{cd})} \hat{J}_{3}^2
    \sum_{abcde} (
    n_a n_b \bar{n}_c \bar{n}_d \bar{n}_e
    + \bar{n}_a \bar{n}_b n_c n_d n_e
    ) \\
    & \quad \left(
    A_{abicde}^{(J_3, J_{ab}, J_{cd})}
    B_{cdeabj}^{(J_3, J_{cd}, J_{ab})}
    - B_{abicde}^{(J_3, J_{ab}, J_{cd})}
    A_{cdeabj}^{(J_3, J_{cd}, J_{ab})}
    \right).
  \end{split}
\end{equation}

The resulting coupled two-body matrix elements are given by the coupled expression
\begin{equation}
  C_{ijkl}^{J_C}
  = D_{ijkl}^{J_C}
  + E_{ijkl}^{J_C}\,,
\end{equation}
with
\begin{align}
  \phantom{D_{ijkl}^{J_C}}
   & \begin{aligned}
    \mathllap{D_{ijkl}^{J_C}} & \equiv
    \frac{1}{6} \frac{1}{\hat{J}_{C}^2}
    \sum_{(J_3, J_{bc})}
    \sum_{abcd}
    (
    n_a \bar{n}_b \bar{n}_c \bar{n}_d
    - \bar{n}_a n_b n_c n_d
    )                                  \\
                              & \quad
    \left(
    A_{ijabcd}^{(J_3, J_C, J_{bc})} B_{bcdkla}^{(J_3, J_{bc}, J_C)}
    - B_{ijabcd}^{(J_3, J_C, J_{bc})} A_{bcdkla}^{(J_3, J_{bc}, J_C)}
    \right),
  \end{aligned} \\
  \phantom{D_{ijkl}^{J_C}}
   & \begin{aligned}
    \mathllap{E_{ijkl}^{J_C}} & \equiv
    (-1)^{j_j + j_l}
    \sum_{(J_{A,3}, J_{ab}, J_{cd})}
    (-1)^{J_{ab} + J_{cd}} \hat{J}_{A,3}^2
    \sum_{J_{B,3}}
    \hat{J}_{B,3}^2                    \\
                              & \quad
    \ninej{j_i}{J_{A,3}}{J_{ab}}{j_j}{J_{cd}}{J_{B,3}}{J_{C}}{j_l}{j_k}
    \sum_{abcd}
    (
    \bar{n}_a \bar{n}_b n_c n_d
    - n_a n_b \bar{n}_c \bar{n}_d
    )                                  \\
                              & \quad
    A_{abicdk}^{(J_{A,3}, J_{ab}, J_{cd})}
    B_{cdjabl}^{(J_{B,3}, J_{cd}, J_{ab})}\,.
  \end{aligned}
\end{align}

The resulting coupled three-body matrix elements are given by the coupled expression
\begin{equation}
  C_{ijklmn}^{(J_{C,3}, J_{C,ij}, J_{C,lm})}
  = D_{ijklmn}^{(J_{C,3}, J_{C,ij}, J_{C,lm})}
  + E_{ijklmn}^{(J_{C,3}, J_{C,ij}, J_{C,lm})}
  + F_{ijklmn}^{(J_{C,3}, J_{C,ij}, J_{C,lm})}\,,
\end{equation}
with
\begin{align}
  \phantom{D_{ijklmn}^{(J_{C,3}, J_{C,ij}, J_{C,lm})}}
   & \begin{aligned}
    \mathllap{D_{ijklmn}^{(J_{C,3}, J_{C,ij}, J_{C,lm})}}
     & \equiv \frac{1}{6} \sum_{J_{ab}}
    \sum_{ab}
    (n_a n_b n_c + \bar{n}_a \bar{n}_b \bar{n}_c)
    \\
     & \quad \left(
    A_{ijkabc}^{(J_{C,3}, J_{C,ij}, J_{ab})}
    B_{abclmn}^{(J_{C,3}, J_{ab}, J_{C,lm})}
    - B_{ijkabc}^{(J_{C,3}, J_{C,ij}, J_{ab})}
    A_{abclmn}^{(J_{C,3}, J_{ab}, J_{C,lm})}
    \right)\,.
  \end{aligned}                         \\
   & \begin{aligned}
    \mathllap{\overline{E}_{ij\bar{n}lm\bar{k}}^{(J_{3},J_{C,ij},J_{C,lm})}} \equiv
    \frac{9}{2} \sum_{J_{ab}}
    \sum_{abc}
    (n_a n_b \bar{n}_c +  \bar{n}_a \bar{n}_b n_c)
    \overline{B}_{ij\bar{n}ab\bar{c}}^{(J_{3}, J_{C,ij}, J_{ab})}
    \overline{A}_{ab\bar{c}lm\bar{k}}^{(J_{3}, J_{ab}, J_{C,lm})}\,,
  \end{aligned}\label{eq:comm333_term2} \\
   & \begin{aligned}
    \mathllap{\overline{F}_{ij\bar{n}lm\bar{k}}^{(J_{3}, J_{C,ij}, J_{C,lm})}}
    \equiv -\frac{9}{2}
    \sum_{J_{ab}}
    \sum_{abc}
    (n_a n_b \bar{n}_c +  \bar{n}_a \bar{n}_b n_c)
    \overline{A}_{ij\bar{n}ab\bar{c}}^{(J_{3}, J_{C,ij}, J_{ab})}
    \overline{B}_{ab\bar{c}lm\bar{k}}^{(J_{3}, J_{ab}, J_{C,lm})}\,.
  \end{aligned}\label{eq:comm333_term3}
\end{align}
The expressions in Eqs.~\eqref{eq:comm333_term2} and~\eqref{eq:comm333_term3}
are written in terms of Pandya-transformed matrix elements,
where the three-body Pandya transformation is given by
\begin{equation}
  \overline{O}_{pq\bar{r}st\bar{u}}^{(J_{3}, J_{pq}, J_{st})}
  \equiv - \sum_{J_{3}'}
  \hat{J}_{3}^{\prime 2}
  \sixj{J_{pq}}{j_r}{J_{3}}{J_{st}}{j_u}{J_{3}'}
  O_{pqustr}^{(J_{3}', J_{pq}, J_{st})}\,.
\end{equation}
The indices $\bar{r}$ and $\bar{u}$ are time reversed,
that is, their angular-momentum projections are flipped.
Since we are working with reduced indices
without angular-momentum projections everywhere,
this distinction is irrelevant,
but we keep the modified indices around to be formally precise.
The Pandya-transformed matrix elements
are useful intermediates that isolate recoupling on $A$, $B$, and $E$/$F$
that can be done independently.
The resulting $\overline{E}$ and $\overline{F}$ must be
converted back to normal coupled matrix elements
by a final Pandya transformation
before being added to $D$ to give $C$.

\subsection{Validation of numerical implementation}

With any numerical implementation,
there is the question of how we can ensure that
the code correctly implements the underlying formalism.
With the $J$-scheme IMSRG,
one has the advantage that
there is the (more transparent) $m$-scheme formalism
that one can compare against.
This means one can uncouple the initial Hamiltonian
fed into the $J$-scheme implementation,
solve the IMSRG(3) flow equations via an $m$-scheme implementation,
and compare the results with those of the $J$-scheme IMSRG(3) solver.
This can be done by looking at the ground-state energy
or the second-order MBPT energy correction
or by recoupling the resulting Hamiltonian matrix elements
and checking that they match the $J$-scheme matrix elements
to within the expected precision.

We take a more fine-grained approach to validating
our $J$-scheme IMSRG(3) implementation,
but using the same overall strategy.
Working with the same initial matrix elements
(in coupled and uncoupled form),
we evaluate the fundamental commutators
(and some many-body operations like
the second-order MBPT energy correction and the two- and three-body antisymmetrizers)
in our $J$-scheme implementation
and an independent $m$-scheme implementation
(used in Section~\ref{sec:imsrg2_he4_mscheme} and Appendix~\ref{app:pairing_hamiltonian_imsrg3}).
The resulting uncoupled matrix elements
provided by the $m$-scheme implementation
are then recoupled and compared with the coupled matrix elements
provided by the $J$-scheme implementation.
With this strategy,
we can demand stringent numerical agreement between the results,
because accumulated numerical error should be low.
We find all tested operations,
which includes all fundamental commutators with antisymmetrization,
the second-order MBPT energy corrections,
and the normal ordering methods,
pass this check.

\section{Outlook towards tensor operators}\label{sec:tensor_jscheme}

Implementing tensor operator support in the IMSRG
provides access to the full range of observables.
Examples of observable tensor operators are
electric and magnetic multipole operators,
which when implemented give access to
low-lying spectroscopy~\cite{Parz17imsrg_em_obs}.

The implementation of tensor operators
is a bit more challenging than that of scalar operators
for a couple of reasons.
First, the coupled matrix elements of tensor operators
are no longer diagonal in total angular momentum $J$
and angular-momentum projection $M$~\cite{Tich20jcoupling}.
The $M$-dependence is taken care of by
the factorization provided by the Wigner-Eckart theorem.
However, when using the reduced matrix elements
the non-diagonality in $J$ remains,
so the representation that worked for scalar operators
needs to be generalized to support this
(along with the inclusion of the tensor rank).

Additionally, one requires a whole new set
of fundamental commutators,
specifically those of a scalar operator
(either the generator $\eta$ or the Magnus operator $\Omega$)
and a tensor operator.
The \texttt{amc} code has no trouble generating
the coupled/reduced expressions for these fundamental commutators~\cite{Tich20jcoupling},
but the implementation effort for these additional commutators
will be similar to (probably greater than) that for the scalar-scalar commutators.
Ultimately,
in order to explore electroweak observables in nuclei,
additional work in this direction must be done
to implement these more general tensor-scalar commutators.
