\chapter*{Acknowledgements}

This work was made possible by many people,
more than I can reasonably mention here,
but I want acknowledge the contributions of the key players here.
I am grateful to my supervisors, Achim Schwenk and Alex Tichai.
Achim is the main reason I came to Darmstadt.
In my time here, he has taught me a lot
in both the classroom and research settings.
He has passed on a lot of knowledge and great intuition,
and his deep knowledge on so many diverse topics
motivates me to broaden my horizons and look beyond my immediate current interests.
Achim also has made a lot of opportunities available to me
during my time as a Master's student.
In particular, I am extremely grateful for the opportunity
to attend summer schools at Michigan State University
and the ECT*.
Alex has been an excellent collaborator on this work,
and without his supervision and feedback
the project would not have come nearly as far.
Alex has brought deep knowledge of nuclear many-body physics
and diverse perspectives to research discussions related to this work,
and I am excited to continue working with him on improving upon the results presented here.

I would also like to thank my collaborators Kai Hebeler and Jan Hoppe,
who also played a supervisory role in my research.
Working with Kai on similarity renormalization group projects
allowed me to initially find solid footing when I arrived in Darmstadt,
and since then he has always been an excellent point of reference
for conversations ranging from work-life balance to high-performance computing,
from nuclear interactions and few-body systems to nuclear matter.
Jan played a crucial role in helping me benchmark many parts of my implementations
and was always available to talk through conceptual and technical problems I ran into.
Jan also generated on fairly short notice nearly all of the matrix elements
used as input in the calculations in this thesis.

In addition to my immediate collaborators,
this work was positively influenced by the stimulating research environment
I am in.
I am grateful to the STRONGINT research group,
where folks are happy to discuss all sorts of topics
and provide an atmosphere that facilitates creative research.
I would like to thank in particular my former and current officemates,
Corbinian, Sabrina, Sebastian, and Victoria,
who were typically the first people I would approach with questions or new ideas.
I would also like to thank the people besides my supervisors
that read through and gave feedback on
parts or all of this thesis,
Jan, Kai, Lars, and Mirko.
Lars and Rodric also provided some useful templates
for the title page and custom bibliography styles
that are used in the thesis.
Beyond the STRONGINT group,
I am grateful to the many pleasant and helpful people
I have gotten to know in the context of the Institut f\"{u}r Kernphysik
and the CRC 1245.
Finally, I am grateful to
Dick Furnstahl and
the OSU low-energy theory group
for helping me develop during my undergraduate time
into a confident and capable nuclear physicist
and continuing to make me feel comfortable whenever I return.

On a personal level,
my path through life and also through the year of this thesis
has only been possible thanks to the support of my parents,
Ulrich Heinz and Christiane Heinz-Neidhart,
and my younger brother,
Michael Heinz.
Of course,
I would be nowhere without my parents,
but in particular their emotional support
during the time of the current pandemic
has enabled me to focus on my work.
I will always be grateful for
how they guided me through life
and the many opportunities they have provided me.
My brother Michael is simultaneously my best friend
and one of my greatest inspirations.
I am glad that I can always look to him for motivation
and turn to him whenever I need advice, help, or just a listening ear.

I am thankful to Stephanie M\"{u}ller
for all of her assistance during the process of moving to Darmstadt
and in the time since.
I am thankful to my siblings,
Jutta Schr\"{o}ter, Richard Heinz, and J\"{u}rgen Dambeck,
for answering my many questions about life in Germany
and how things work.

Last but not least,
I would like to thank my friends.
I value the diverse set of perspectives and interests
they bring to my life,
and I appreciate that we are able to have uninhibited fun
when we spend time together.

This work is supported in part by the Deutsche Forschungsgemeinschaft
(DFG, German Research Foundation) – Projektnummer 279384907 – SFB 1245.
