\chapter{Simplification strategies for coupled fundamental commutators}\label{app:jscheme_commutator_tricks}

For the coupled fundamental commutators,
the key to arriving at the simplest expressions
is to avoid unnecessary recoupling.
Recoupling,
where bra or ket couplings have to be broken and recreated
to relate matrix elements on both sides of the expression,
is unnecessary when symmetries of the matrix elements
or of the expression can be exploited
to bring the uncoupled expression into a form
where its coupled analog no longer requires recoupling.
Recoupling comes at the cost of an additional sum over an angular momentum
and the inclusion of a Wigner $6j$ or $9j$ symbol.
Thus avoiding unnecessary recoupling is essential for achieving better performance
for implementations of the coupled expressions.
In the following,
we discuss two strategies for avoiding recoupling
and apply them each to one example.

\section{Using antisymmetry of input operator matrix elements}

The uncoupled matrix elements are antisymmetric under exchange
of any two bra or ket indices.
This symmetry can easily be exploited to permute indices into a form
that minimizes recoupling.

Consider the $[\onebodyop{A}, \threebodyop{B}] \rightarrow \twobodyop{C}$ fundamental commutator.
The uncoupled expression for the matrix elements of $C$ is
\begin{equation}
    C_{ijkl} = \sum_{ab} (n_a - n_b) A_{ab} B_{bijakl}\,,
\end{equation}
and the coupled expression is
\begin{equation}
    \begin{split}
        C_{ijkl}^{J_C} &= (-1)^{j_i + j_j + j_k + j_l}
        \sum_{j_a} \sum_{(J_{B,3}, J_{B,pq}, J_{B,st})} \hat{J}_{B,pq}\hat{J}_{B,st}\hat{J}_{B,3}^2 \\
        &\quad\sum_{ab} (n_a \bar{n}_b - \bar{n}_a n_b)
        \sixj{j_j}{j_i}{J_C}{j_a}{J_{B,3}}{J_{B,pq}}
        \sixj{j_l}{j_k}{J_C}{j_a}{J_{B,3}}{J_{B,st}} \\
        &\quad A_{ab}^{j_a} B_{bijakl}^{(J_{B,3}, J_{B,pq}, J_{B,st})}\,.
    \end{split}
\end{equation}

The appearance of 6$j$ symbols is a result of
breaking the $bi$ coupling on the right to form the $ij$ coupling on the left
and breaking the $ak$ coupling on the right to form the $kl$ coupling on the left.
By exploiting antisymmetry in the uncoupled matrix elements of $B$,
we can arrive at the uncoupled expression
\begin{equation}
    C_{ijkl} = \sum_{ab} (n_a \bar{n}_b - \bar{n}_a n_b) A_{ab} B_{ijbkla}\,,
\end{equation}
which yields the coupled expression
\begin{equation}
    C_{ijkl}^{J_C} = \frac{1}{\hat{J}_{C}^2}
    \sum_{j_a}
    \sum_{J_{B,3}} \hat{J}_{B,3}^2
    \sum_{ab} (n_a \bar{n}_b - \bar{n}_a n_b)
    A_{ab}^{j_a} B_{ijbkla}^{(J_{B,3}, J_{C}, J_{C})}\,.
\end{equation}

\section{Using antisymmetry of output operator matrix elements}

As a reminder, in this work we typically work with
non-antisymmetrized expressions for the fundamental commutators,
so one can not simply permute the indices on the output matrix elements.
However, one can exploit that the output matrix elements
will at some point be antisymmetrized
and choose the term that contributes to the antisymmetrized result
with the minimal recoupling.

To make this clear, consider the following example:
Suppose the three-body uncoupled matrix elements $O_{pqrstu}$
are already antisymmetric in $stu$, but not in $pqr$.
Thus to get the final antisymmetrized matrix elements $O^{\text{as}}_{pqrstu}$,
one needs to evaluate
\begin{equation}\label{eq:antisymmetrize_short}
    O^{\text{as}}_{pqrstu} = \mathcal{A}_{pqr} O_{pqrstu}\,,
\end{equation}
with
\begin{equation}
    \mathcal{A}_{pqr} = \frac{1}{6} (1 + P_{prq} + P_{prq}^2) (1 - P_{pq})\,,
\end{equation}
where $P_{prq}$ cyclically permutes the indices $p$, $q$, and $r$
and $P_{pq}$ exchanges the indices $p$ and $q$
as discussed in Section~\ref{sec:jscheme_many_body_expressions}.

Writing Eq.~\eqref{eq:antisymmetrize_short} out, one gets
\begin{equation}\label{eq:antisymmetrize_long_normal}
    O^{\text{as}}_{pqrstu} = \frac{1}{6} (O_{pqrstu} + O_{qrpstu} + O_{rpqstu} - O_{qprstu} - O_{rqpstu} - O_{prqstu})\,.
\end{equation}
Now consider
\begin{equation}
    \overline{O}_{pqrstu} \equiv O_{qrpstu}\,.
\end{equation}
Antisymmetrizing $\overline{O}_{pqrstu}$ gives
\begin{equation}
    \overline{O}^{\text{as}}_{pqrstu} = \frac{1}{6} (\overline{O}_{pqrstu} + \overline{O}_{qrpstu} + \overline{O}_{rpqstu}
    - \overline{O}_{qprstu} - \overline{O}_{rqpstu} - \overline{O}_{prqstu})\,,
\end{equation}
which one can rewrite in terms of the original matrix elements $O_{pqrstu}$:
\begin{equation}\label{eq:antisymmetrize_long_alternate}
    \overline{O}^{\text{as}}_{pqrstu} = \frac{1}{6} (O_{qrpstu} + O_{rpqstu} + O_{pqrstu}
    - O_{rqpstu} - O_{prqstu} - O_{qprstu})\,.
\end{equation}
One can quickly see that Eqs.~\eqref{eq:antisymmetrize_long_normal} and~\eqref{eq:antisymmetrize_long_alternate}
contain the exact same terms, making them equal.
Thus if $O_{pqrstu}$ is given by some many-body expression,
one can permute the indices $p$, $q$, and $r$ on the right-hand (or left-hand) side of the expression,
including an overall minus sign if the permutation is odd,
and know that it will give the same result after antisymmetrization.

Again, it is useful to consider a concrete application to illustrate how this strategy can be applied.
Consider the $[\twobodyop{A}, \twobodyop{B}] \rightarrow \threebodyop{C}$ fundamental commutator.
The uncoupled non-antisymmetrized expression is
\begin{equation}
    C_{ijklmn} = 9 \sum_{a} (A_{ijla} B_{akmn} - B_{ijla} A_{akmn})\,.
\end{equation}
which gives the coupled expression
\begin{equation}\label{eq:comm_223_bad}
    \begin{split}
        C_{ijklmn}^{(J_{C,3}, J_{C,ij}, J_{C,lm})} & =
        9 \hat{J}_{C,ij} \hat{J}_{C,lm}
        (-1)^{j_m + j_n}
        \sum_{J_{mn}}
        \hat{J}_{mn}^2
        \sum_{a} (-1)^{j_a + j_k} \\
        & \quad \quad \times
        \sixj{j_k}{j_a}{J_{mn}}{j_{l}}{J_{C,3}}{J_{C,ij}}
        \sixj{j_l}{j_m}{J_{C,lm}}{j_{n}}{J_{C,3}}{J_{mn}} \\
        & \quad \quad \times \left(
        A_{ijla}^{J_{C,ij}} B_{akmn}^{J_{mn}}
        - B_{ijla}^{J_{C,ij}} A_{akmn}^{J_{mn}}
        \right).
    \end{split}
\end{equation}
One can immediately see that
the $mn$ coupling on the right in $B$ and $A$ needs to be broken up
to form the $lm$ coupling on the left,
leading to some of the recoupling.

However, the uncoupled expression
\begin{equation}
    D_{ijklmn} \equiv 9 \sum_{a} (A_{ijna} B_{aklm} - B_{ijna} A_{aklm})
\end{equation}
has the indices $l$, $m$, and $n$ on the right-hand side permuted cyclically,
and it will no longer have this recoupling problem when the expression is coupled.
The antisymmetrized matrix elements of $C$ and $D$ are equal,
so one might as well simply calculate $D$ instead of $C$ and antisymmetrize that.
The coupled expression for $D$ is
\begin{equation}\label{eq:comm_223_good}
    \begin{split}
        D_{ijklmn}^{(J_{C,3}, J_{C,ij}, J_{C,lm})} & =
        -9 (-1)^{J_{C,lm}} \hat{J}_{C,ij} \hat{J}_{C,lm}
        \sum_{a} (-1)^{j_a + j_k}
        \sixj{j_k}{j_a}{J_{C,lm}}{j_{n}}{J_{C,3}}{J_{C,ij}} \\
        & \quad \quad \times \left(
        A_{ijna}^{J_{C,ij}} B_{aklm}^{J_{C,lm}}
        - B_{ijna}^{J_{C,ij}} A_{aklm}^{J_{C,lm}}
        \right).
    \end{split}
\end{equation}
The final $6j$ symbol cannot be avoided
as it forms the three-body angular-momentum coupling
that is not present on the right-hand side.
Comparing Eq.~\eqref{eq:comm_223_bad} and Eq.~\eqref{eq:comm_223_good},
one can see the simplification afforded by carefully considering
which quantity one actually wants to calculate.
