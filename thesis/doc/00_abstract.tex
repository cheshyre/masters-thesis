\chapter*{Abstract}

The in-medium similarity renormalization group (IMSRG)
is an \abinitio{} many-body method used to great success
to solve the time-independent Schr\"{o}dinger equation in medium-mass nuclear systems.
Its computational cost scales polynomially in the size of the truncated model space,
and its formalism is highly flexible,
leading to multiple variants that have been developed to extend its original closed-shell
formulation to open-shell systems.

The current state-of-the-art implementations truncate the IMSRG equations
at the normal-ordered two-body level,
the first non-trivial order in the expansion.
In this work, we seek to systematically study the effects
of extending this truncation to the normal-ordered three-body level,
the so-called IMSRG(3) approximation.
Exploitation of symmetries is essential to making IMSRG(3) calculations tractable.
We present the reduced $J$-scheme IMSRG(3) working equations,
which we arrive at by applying angular-momentum reduction to the IMSRG(3) for spherical systems.

We use our implementation of the $J$-scheme IMSRG(3)
to investigate three-body contributions
that first appear in the IMSRG(3)
in light and medium-mass nuclei.
We introduce approximate IMSRG(3) truncations
that leave out the most expensive parts of the IMSRG(3).
We find that in ${}^{4}\text{He}$
and ${}^{16}\text{O}$
in a restricted $\emax=2$ model space,
these approximate IMSRG(3) truncation schemes
deliver small, sub-percent corrections to the ground-state energies
and larger corrections to radii.
Further, by investigating the behavior
under the removal or inclusion of certain terms,
we see that the organization by computational cost
used to set up our approximate truncation schemes
is poorly motivated
and some computationally more expensive terms
provide larger corrections to ground-state energies
than the cheaper terms in the truncation.
This work is a key step towards
high-precision many-body calculations of medium-mass nuclei
in the IMSRG.

