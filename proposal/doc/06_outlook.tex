\chapter{Summary and outlook}\label{ch:summary}

In this thesis, we aim to study the ground-state properties of closed-shell nuclear systems in the IM-SRG(3).
As a first step in that direction, we have implemented a generic IM-SRG(2)/(3) solver,
which we applied at the IM-SRG(2) truncation level to the calculation of the ${}^4\text{He}$ ground-state energy
and at the IM-SRG(2) and IM-SRG(3) level to the pairing Hamiltonian.
At the IM-SRG(2) level, we found excellent agreement between our implementation
and existing IM-SRG(2) implementations and published results~\cite{Stro15imsrgcpp,Herg16imsrglecnotes}.
For the pairing Hamiltonian at the IM-SRG(3) level,
we found that the three-body truncation does reduce the discrepancy between IM-SRG(2) and the exact solution
in the region of large coupling.
This can be interpreted as evidence that the many-body expansion is behaving systematically,
a claim that we would also like to verify in nuclear systems.

In principle, the next step would be to calculate ${}^4\text{He}$
with the same Hamiltonian in the IM-SRG(3),
where one would expect the corrections due to the three-body truncation to be small.
With our $\emax=2$, this amounts to solving a system of approximately 4 billion coupled differential equations,
most of which are the flowing matrix elements of the three-body Hamiltonian $W$.
Of course, many of these matrix elements are not independent,
related to one another via antisymmetry or (anti-)Hermiticity,
and many should be exactly equal to 0,
such as those that are blocked by the Pauli exclusion principle.
By exploiting these symmetries, it may be possible to allow the current implementation
of IM-SRG(3) to also work for ${}^4\text{He}$.
This would offer the ability to look at many different metrics,
such as the flowing ground-state energy and the MBPT corrections,
when implementing a $J$-scheme IM-SRG(3) library.
Even if the performance is insufficient, a calculation can be run for a few integration steps
and the implementation of the fundamental commutators can be used
to benchmark the $J$-scheme implementation of the fundamental commutators.

Looking forward to the next phase of the project,
some key deliverables will be:
\begin{enumerate}
  \item{
    implementing a correct $J$-scheme IM-SRG(3) solver
    for use in nuclear many-body calculations
    using automatic recoupling tools~\cite{Tich20jcoupling};
  }
  \item tuning the performance of the solver to reach single-particle basis sizes large enough
    for observables to be reasonably converged;
  \item calculating the ground-state energies (and possibly other observables) for ${}^4\text{He}$
    and ${}^{16}\text{O}$ in the IM-SRG(3) and systematically studying the effects of the three-body truncation.
\end{enumerate}
It will be interesting to see if the improvement looks significantly different for different Hamiltonians
and if there are some well-reasoned physically-motivated ways to approximate the IM-SRG(3) well
that could be used to extend current large-scale IM-SRG(2) implementations.
Some options for such approximations might be
the inclusion of certain terms that dominate the IM-SRG(3) contributions
or the development of a way to include sparse approximate three-body operators
that provide the dominant IM-SRG(3) contributions.
With a robust implementation of our own,
we will be in an excellent position to systematically study these options and many more.

